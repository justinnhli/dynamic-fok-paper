\documentclass[10pt,letterpaper]{article}

\usepackage{cogsci}
\usepackage{apacite}

\usepackage{amsmath}
\usepackage{amsfonts}
\usepackage[usenames,dvipsnames]{color}
\usepackage{graphicx}
\usepackage{microtype}
\usepackage[normalem]{ulem}

\newcommand{\setof}[1]{\left \{ #1 \right \}}
\newcommand{\tuple}[1]{\left \langle #1 \right \rangle }

\hyphenpenalty=1500
\relpenalty=9999
\binoppenalty=9999
\hbadness=3500
\vbadness=2000
\frenchspacing

\setlength{\pdfpagewidth}{8.5in}
\setlength{\pdfpageheight}{11in}
%\setlength{\abovecaptionskip}{0.5em}

\newcommand{\comment}[1]{}

% \fixme[comments describing issue]{text to be fixed}
% define \fixme as not doing anything special
\newcommand{\fixme}[2][]{#2}
% overwrite it so it shows up as red
\renewcommand{\fixme}[2][]{\textcolor{red}{#2}}
% overwrite it again so related text shows as footnotes
%\renewcommand{\fixme}[2][]{\textcolor{red}{#2\footnote{#1}}}

\newcommand{\changed}[1]{#1}
%\renewcommand{\changed}[1]{\textcolor{blue}{#1}}

\comment{
    \pdfinfo{
        /Title (Towards a Computational Model of Feeling of Knowing)
        /Subject (Proceedings of the 18\textsuperscript{th} International Conference on Cognitive Modeling)
        /Author (Justin Li, Bryce Boyle)
        /Keywords (Feeling of Knowing)
    }

    %\title{Towards a Computational Model of Feeling of Knowing}
    
    \author{
        {\large \bf Justin Li (justinnhli@oxy.edu)} \\
        {\large \bf Emma Kohanyi (kohanyi@oxy.edu)} \\
        Occidental College, 1600 Campus Road \\
        Los Angeles, CA 90041 USA
    }
}

\pdfinfo{
    /Title (Towards a Computational Model of Feeling of Knowing)
    /Subject (Proceedings of the 18\textsuperscript{th} International Conference on Cognitive Modeling)
    /Author (Justin Li, Bryce Boyle)
    /Keywords (Feeling of Knowing)
}

\title{Towards a Computational Model of Feeling of Knowing}

\author{
    {\large \bf Anonymized}
}

\begin{document}

\maketitle

\begin{abstract}

    \fixme{Lorum ipsum}

    \textbf{Keywords:} Feeling of Knowing, Knowledge Search

\end{abstract}

\section{Introduction}

\emph{Feelings of knowing} (FOKs) are a memory phenomenon in psychology where, despite not knowing the answer, the subject might \emph{feel} that they will be able to retrieve the answer with more thought.
This and other metamemory phenomena, including \emph{tip-of-the-tongue states} and \emph{judgments of learning}, have long been studied as both a topic area of its own, as well as a way to gain insight into how memory is related to decision making \cite{Nelson1994WhyInvestigateMetacognition}.
The psychology literature have focused on both the situations that lead to changes in FOKs, as well as the underlying determinants of the judgment.
More recently, researchers have proposed the \emph{cognitive-heuristic account} of metamemory: that FOK serves the \emph{function} of guiding memory retrieval \cite{Schwartz2011TipOfThe}, and to allow early failure without expending retrieval resources if the probability of finding an answer is low.

In parallel, the cognitive modeling community interested in modeling memory since the founding of the subfield.
Cognitive architectures like ACT-R \cite{Anderson2007HowCanThe} in part grew out of early models of declarative memory \cite{Anderson1983ASpreadingActivation, Anderson1991ReflectionsOfThe}, and to this day the mechanisms of base-level activation and spreading activation remain deeply embedded within multiple architectures.
While recognition and recall have been widely modeled, however, metamemory phenomena remain understudied, especially in the context of the broader strategic knowledge search \cite{Newell1990UnifiedTheoriesOf}.
This paper therefore aims to complement existing literature by exploring how metacognitive judgment and control, and feelings of knowing in particular, might be modeled in a common model of cognitive architecture.
We begin by reviewing the literature on strategic knowledge search in order to place FOK in its ecological context, and identify a potential omission in our current understand of FOK.
We then briefly summarize the representation of memory in ACT-R, before committing the bulk of this paper to considering different possible determinants of a computational FOK.
Finally, we conclude by discussing how such a model of FOK might be evaluated, especially given the subjective nature of such judgments in the psychological literature.

\section{Strategic Knowledge Search}



\section{Modeling Definitions}

Formally, the contents of LTM is an edge-labeled directed graph, defined by:

\begin{description}
    \item[$S$]: the set of entities, which corresponds to the internal nodes of the graph;
    \item[$P$] : the set of predicates, which corresponds to the edge labels of the graph;
    \item[$L$] : the set of literals, such as numbers and strings, which corresponds to the leaf nodes of the graph;
    \item[$E$] : the set of \emph{triples} $\tuple{S, P, S \cup L}$, which we will refer to as the \emph{subject}, the \emph{predicate}, and the \emph{object} respectively. These correspond to the directed edges of the graph.
\end{description}

For convenience, we will denote the set of possible objects as $O = S \cup L$.

\section{Determinants of Feelings of Knowing}

\section{Caveats}

\section{General Discussion}


% PLACEHOLDER TO TRIGGER CITATIONS
%
% \cite{Burgess1996ConfabulationAndThe}
% \cite{DeSoto2014PositiveAndNegative}
% \cite{Florer2000FeelingsOfKnowing}
% \cite{Hanczakowski2014FeelingOfKnowing}
% \cite{Hanczakowski2017MetamemoryInA}
% \cite{Koriat1993HowDoWe}
% \cite{Koriat2001TheCombinedContributions}
% \cite{Li2012FunctionalInteractionsBetween}
% \cite{Li2016ArchitecturalMechanismsFor}
% \cite{Mangan2000WhatFeelingIs}
% \cite{Nelson1984AccuracyOfFeeling}
% \cite{Nelson1990MetamemoryATheoretical}
% \cite{Nelson2008TowardsARational}
% \cite{Salvucci2015EndowingACognitive}
% \cite{Schreiber1998TheRelationBetween}
% \cite{Schwartz2011TipOfThe}
% \cite{Schwartz2014ContextualInformationInfluences}
% \cite{Sharma2016ControllingSearchIn}
% \cite{Widner1996TheEffectsOf}
% \cite{Norman2016TheRelationshipBetween}

\bibliographystyle{apacite}

\setlength{\bibleftmargin}{.125in}
\setlength{\bibindent}{-\bibleftmargin}

\bibliography{iccm}

\end{document}
