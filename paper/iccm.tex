% 
% Annual Cognitive Science Conference
% Sample LaTeX Paper -- Proceedings Format
% 

% Original : Ashwin Ram (ashwin@cc.gatech.edu)       04/01/1994
% Modified : Johanna Moore (jmoore@cs.pitt.edu)      03/17/1995
% Modified : David Noelle (noelle@ucsd.edu)          03/15/1996
% Modified : Pat Langley (langley@cs.stanford.edu)   01/26/1997
% Latex2e corrections by Ramin Charles Nakisa        01/28/1997 
% Modified : Tina Eliassi-Rad (eliassi@cs.wisc.edu)  01/31/1998
% Modified : Trisha Yannuzzi (trisha@ircs.upenn.edu) 12/28/1999 (in process)
% Modified : Mary Ellen Foster (M.E.Foster@ed.ac.uk) 12/11/2000
% Modified : Ken Forbus                              01/23/2004
% Modified : Eli M. Silk (esilk@pitt.edu)            05/24/2005
% Modified : Niels Taatgen (taatgen@cmu.edu)         10/24/2006
% Modified : David Noelle (dnoelle@ucmerced.edu)     11/19/2014
% Modified : Roger Levy (rplevy@mit.edu)             12/31/2018

\documentclass[10pt,letterpaper]{article}

\usepackage{cogsci}
%\cogscifinalcopy % Uncomment this line for the final submission 
\usepackage{pslatex}
\usepackage{apacite}
% Roger Levy added this and changed figure/table
% placement to [H] for conformity to Word template,
% though floating tables and figures to top is
% still generally recommended!
\usepackage{float}

\usepackage{amsmath}
\usepackage{amsfonts}
\usepackage[usenames,dvipsnames]{color}
\usepackage{graphicx}
\usepackage{microtype}
\usepackage{multirow}
\usepackage[normalem]{ulem}
\usepackage[table]{xcolor}

\hyphenpenalty=1500
\relpenalty=9999
\binoppenalty=9999
\hbadness=3500
\vbadness=2000
\frenchspacing

\newcommand{\comment}[1]{}

% usage: \fixme[comments describing issue]{text to be fixed}
% define \fixme as not doing anything special
\newcommand{\fixme}[2][]{#2}
% overwrite it so it shows up as red
\renewcommand{\fixme}[2][]{\textcolor{red}{#2}}
% overwrite it again so related text shows as footnotes
%\renewcommand{\fixme}[2][]{\textcolor{red}{#2\footnote{#1}}}

\newcommand{\textcite}[1]{\citeauthor{#1} \citeyear{#1}}

\newcommand{\negsub}[0]{\;\!^{\_}}

\newcommand{\tableheader}[1]{\multirow{2}{*}{\textbf{#1}}}
\newcommand{\whitecell}[0]{}
\newcommand{\graycell}[0]{\cellcolor{gray!70}}
\newcommand{\blackcell}[0]{\cellcolor{black}}

\newcommand{\setof}[1]{\left \{ #1 \right \}}
\newcommand{\tuple}[1]{\left \langle #1 \right \rangle }
\newcommand{\fok}[0]{\text{FOK}}

\newcommand{\question}[2]{\textit{#1}\footnote{Answer: #2}}


%\title{Towards a Dynamic Feeling of Knowing and its Cognitive Model}

%\author{
%    {\large \bf Justin Li (justinnhli@oxy.edu)} \\
%    {\large \bf Bryce Boyle (bboyle@oxy.edu)} \\
%    Occidental College, 1600 Campus Road \\
%    Los Angeles, CA 90041 USA
%}

\title{Towards a Computational Model of Feeling of Knowing that Changes over Time} % FIXME

\author{
    {\large \bf Anonymized}
}

\begin{document}

\maketitle

\begin{abstract}

    Feelings of knowing (FOKs) are metamemory judgments that suggest an answer could be retrieved from memory with more effort.
    This paper reviews the psychology literature on what factors underlie FOKs and map them onto the memory mechanisms of the common model of cognition architecture.
    The two \fixme{most common} accounts of FOK, that of cue familiarity and accessibility, map directly onto properties of the retrieval cue and result respectively.
    In considering these models of FOK, we also \fixme{highlight} \fixme{an omission} from the literature: the possibility that FOK changes over the course of answering a question.
    We conclude by discussion the implications of this dynamic account, as well as the difficulties of evaluating computational models of FOK.

    \textbf{Keywords:} Feeling of Knowing; Metamemory; Knowledge Search; Cognitive Architecture

\end{abstract}

% FIXME should pick either "sources" or "determinants"

\section{Introduction}

\textit{Feelings of knowing} (FOKs) are a memory phenomenon where, despite not retrieving the answer to a question in the moment, the subject \textit{feels} that they will be able to do so with more effort.
This and other metamemory phenomena have been studied as a topic of its own and as a way to gain insight into how memory is used to decision making \cite{Nelson1994WhyInvestigateMetacognition}, with research focusing on the determinants of FOK and what situational factors influence the metamemory judgment.
More recently, researchers have proposed the \textit{cognitive-heuristic account} of metamemory: that FOK serves the function of guiding memory retrieval, allowing for early failure without expending resources if the probability of finding an answer is low \cite{Schwartz2011TipOfThe}.
At the same time, the cognitive modeling community has been interested in modeling memory since its founding.
Cognitive architectures such as ACT-R \cite{Anderson2007HowCanThe} grew out of early models of declarative memory, and to this day the mechanisms of base-level activation and spreading activation remain deeply embedded.
Unlike recognition and recall, however, metamemory phenomena remain understudied, especially in the context of broader strategic knowledge search \cite{Newell1972HumanProblemSolving}.

This paper therefore aims to complement existing literature by exploring how feelings of knowing might be instantiated in a common model of cognitive architecture \cite{Laird2017AStandardModel}.
We begin by placing FOKs in the context of knowledge search, and in doing so identify a potential omission in our current understand of the phenomenon.
We then briefly summarize the representation of memory in ACT-R, before committing the bulk of this paper to considering different possible sources of a computational FOK.
Finally, we conclude by discussing how such a model of FOK might be evaluated, especially given the subjective nature of such judgments in the psychological literature.

\section{FOK and Strategic Knowledge Search}

As a \fixme{phenomenon} of memory, FOKs must be understood in the context of how it may help an agent retrieve knowledge; that is, in the context of \textit{knowledge search}.
First proposed by \textcite{Newell1972HumanProblemSolving}, knowledge search is the process of finding knowledge that is relevant and could be applied to the current problem solving context.
\citeauthor{Newell1972HumanProblemSolving} do not elaborate on the processes of knowledge search, and knowledge search has received scant exploration in contrast to problem space search.
However, there has been a trickle of psychological research on how people use memory in naturalistic settings, and on the problem solving and decision processes that are intertwined with memory.
The results show a rich landscape of memory processes: beyond basic recognition and recall, subjects described determining the recall specifications, gauging their own familiarity with the subject, relating multiple relevant memories, and verifying that a potential answer is in fact correct \cite{Burgess1996ConfabulationAndThe}.
This account validates the idea that metamemory judgments such as FOK are used for control of retrieval processes, such as selecting a search strategy and deciding whether to terminate search \cite{Nelson1990MetamemoryATheoretical}.
More recently, FOK has been shown to influence search termination and decision of what to rehearse \cite{Singer2008FeelingOfKnowing,Hanczakowski2014FeelingOfKnowing}, further corroborating the cognitive-heuristic account of metamemory.

It is clear that FOKs must be understood within this broader context of knowledge search.
If FOK is to guide memory retrieval, it must provide meaningful information about the state and contents of memory, such that an agent could determine the next retrieval from memory. % FIXME something about signal to noise ratio here?
However, this raises questions about our understanding of FOK.
Thus far, experimental procedures for FOKs tend to only solicit a single judgment, either before or after the subjects attempt to retrieve the answer.
At the same time, our understanding of strategic memory processes suggests that multiple retrievals are often necessary to return an answer, with each individual retrieval potentially giving irrelevant results. % FIXME cite examples of cognitive architectures doing this?
As an example, for a question like \question{What film was nominated for seven Academy Awards in 1999?}{\textit{Life is Beautiful}} \cite{Norman2016TheRelationshipBetween}, one might perform retrievals for general cultural events in 1999, famous directors and actors/actresses, generally acclaimed films, and so on.
This account complicates the study of FOK: about \textit{which} retrieval is the subject feeling that they know the answer to?
When a subject reports their feeling of knowing, is it to the original question, or to any of the sub-questions that they ask themselves as they engage in the strategic search for the answer?

Here we propose that the reported FOK is to the original question, and not to any of the other retrievals for the search process.
This \fixme{conclusion} is more obvious for a question such as \question{What is the capital of Australia?}{Canberra}.
Most people will suggest answers such as Sydney, Melbourne, and Brisbane before giving up, but despite these successful retrievals for Australian cities, will report that their FOK goes down over time before they terminate their search.
That is, it seems clear to us that FOK is a \textit{dynamic} signal that changes throughout the strategic memory search process: as additional retrievals are used for problem solving, the FOK fluctuates for the overarching goal of answering the original question.
This is also consistent with the cognitive-heuristic account of FOK: in order for FOK to be a reliable signal for search termination, it \textit{must} change over the course of the process to reflect whether an answer is still likely to be found.
This stands in contrast to how FOK is usually studied: all psychological experiments we have found only solicit subjects' FOKs once, either before or after they are given the chance to attempt to answer the question. % FIXME are there studies that hint at the dynamic nature of FOK? yes - Florer2000FeelingsOfKnowing, Koriat2001TheCombinedContributions, https://link.springer.com/article/10.3758/s13423-021-01930-z
While we have no doubt that such reports of FOK will still be correlated with the state of memory, ignoring the time course of FOK will likely omit crucial aspects of how the signal is determined.
For the remainder of this paper, we will therefore assume this dynamic view of FOK as we consider how it might be modeled computationally.

\section{Memory in the Common Model of Cognition}

We now describe the agent framework in which we are modeling FOK, namely, that of the common model of cognition \cite{Laird2017AStandardModel}.
The common model defines a set of representations and processes for modeling cognition, as implemented in cognitive architectures such as ACT-R and Soar \cite{Anderson2007HowCanThe,Laird2012TheSoarCognitive}.
Of particular interest to this paper are the declarative long-term memory (LTM) processes, specifically that of semantic memory, which we describe below.

Formally, the contents of LTM is an edge-labeled directed graph, defined by the tuple $\tuple{S, P, L, E}$:
$S$ the set of entities or concepts (we use these terms interchangeably), which corresponds to the internal nodes of the graph;
$P$ the set of predicates, which corresponds to the edge labels of the graph;
$L$ the set of literals, such as numbers and strings, which corresponds to the leaf nodes of the graph;
and $E$ the set of direct edges, $\tuple{s, p, o} \subset S{\times}P{\times}O$, with $O=S{\cup}L$.
Borrowing from the knowledge representation literature, we will also refer to edges as \textit{triples}, and refer to the elements of a triple $\tuple{s, p, o}$ as the \textit{subject}, the \textit{predicate}, and the \textit{object} respectively.

An agent has two ways of getting knowledge from LTM.
First, for any entity $s$, the agent can \textit{retrieve} all outgoing edges $\setof{\tuple{s, p, o}{\in}E}$ for which that entity is the subject.
This mechanism is for accessing related information of a known concept, but to find an unknown concept that has certain properties, the agent must \textit{query} LTM instead.
To do so, the agent creates a query \textit{cue} $Q = \setof{q{\in}P{\times}O}$, which describes the predicates and corresponding objects of the desired entity $s$ such that $S_Q = \setof{\tuple{s, p, o}{\in}E \; \forall \tuple{p, o}{\in}Q}$.
If more than one such \textit{retrieval candidate} exists, the entities with higher \textit{base-level activation} are preferentially returned.
Base-level activation is determined by $A(s) = \ln (\sum {t_i^{-d}})$, where $t_i$ is the time since the entity $s$ was last retrieved, and $d$ is a decay rate parameter.
Activation thus captures the recency and frequency of use of a concept, and is often used as a proxy of the importance of the concept to the agent at a particular time.
% FIXME consider talking about Companions, where the "cue" is the analogy target \cite{Forbus1995MACFACAModel}

Within this framework, we can define the general process through which an FOK might be generated.
When the agent is presented with a question, the agent would execute a sequence of queries and retrievals to LTM to attempt to answer the question.
For clarity, we call the answer to the overarching question the \textit{target} of the question, while an individual query will have a \textit{result} (the entity that is returned) out of a set of \textit{candidates} (other entities that match the cue).
We assume that the FOK for the original question will change with each query and retrieval, and we are therefore interested in the computational processes that occur at those times and how they might affect the overall FOK.

A quick note on terminology: the term \textit{retrieval} is overloaded in both psychology and cognitive architecture literature to sometimes mean both queries (with a cue) and retrievals (of a known concept in LTM).
Retrieval will be used in the psychological sense in this paper; we will disambiguate the term as needed when talking about the specific computational mechanism.

% FIXME something about FOK as a by-product/observer process
% as opposed to a different system separate from memory
% Koriat1993HowDoWe

\section{Psychological Accounts of FOK}

This section explores how psychological accounts of FOKs might be realized within the common model of cognition.
Within the psychology literature, there are three main accounts of FOK: cue familiarity, accessibility, and competition.
For each, we first discuss the relevant psychological literature, before evaluating how that may be translated computationally into a common model LTM.
Since the literature primarily assumes a static FOK for a question, instead of changing over time, these computational models are all calculated from a single retrieval.
A summary of these sources of information for FOK in this section can be found in Table \ref{sources}.

Two mathematical caveats must be considered.
First, FOK may be a function of multiple values; we will assume that it is a summary statistic denoted as $f()$.
Although the choice of summary statistic may affect the FOK calculation --- the mean will be more sensitive to outliers than the median, for example --- we consider this detail too low level for this paper.
It is an open question whether other properties of the distribution, such as its variance, play a role in FOK. % FIXME
Second, it is unclear what the range of the summary statistic function should be.
The main constraint is the ability of the agent to determine whether an FOK judgment is high or low and thereby make retrieval decisions.
The output could theoretically range over the real numbers --- such as if FOK was the activation of a concept --- where decision thresholds are learned over time. % FIXME reword
As we consider a dynamic FOK that may shift between difference sources of information, however, normalizing the FOK may be necessary.
One method for doing so is to consider the \fixme{FOK source} against some distribution of values.
Since the \fixme{comparison} population depends on the source, we will consider an appropriate comparison group for each source. % FIXME

% We leave the evaluation of these models to its own section afterwards. % FIXME revisit when the conclusion is written

\begin{table*}[ht]
\centering
\begin{tabular}{c|cccccc}
\whitecell           &  \tableheader{Cue}  &  \textbf{Cue}           &  \tableheader{Candidates}  &  \textbf{Candidate}      &  \tableheader{Result}  &  \textbf{Result}        \\  
\whitecell           &  \whitecell         &  \textbf{Neighborhood}  &  \whitecell                &  \textbf{Neighborhoods}  &  \whitecell            &  \textbf{Neighborhood}  \\  
\hline                                                                                                                                                                              
\textbf{Count}       &  \graycell          &  Cue Familiarity        &  Accessibility             &  \whitecell              &  \graycell             &  Accessibility          \\  
\textbf{1/Count}     &  \graycell          &  \whitecell             &  Competition               &  \whitecell              &  \graycell             &  Competition            \\  
\textbf{Activation}  &  \blackcell         &  Cue Familiarity        &  Competition               &  \whitecell              &  Accessibility         &  Accessibility          \\  
\end{tabular}

\caption{
    Difference sources of information for calculating FOK, and which corresponding psychology FOK theory they belong in.
    ``Neighborhood'' refers to the concepts that are graph neighbors of the cue, candidate, or result.
    Black cells represent sources that do not exist (e.g., the activation of the retrieval cues, if the cues do not exist in LTM);
    gray cells represent sources that exist but not meaningful for FOK (e.g., the number of retrieval results, which is constant);
    and white cells represent sources that exist and may be meaningful, but have not been explored in the literature.
    \fixme[Need to explain ``neighborhood'']{}
}
\label{sources}
\end{table*}

\subsection{Cue Familiarity}

As the name implies, the \textit{cue familiarity} account of FOK focuses on the contribution of the search cue to the feeling of knowing \cite{Reder1992WhatDeterminesInitial,Metcalfe1993TheCueFamiliarity,Koriat2001TheCombinedContributions}.
For the purpose of this paper, we include all FOK sources that are based on the cue, including familiarity and domain knowledge \cite{Schwartz1994SourcesOfInformation}.
The intuition is that FOK is a summary of the amount of knowledge the agent might have about the topic, as estimated from the terms of the question.
The more the agent is familiar or knowledgeable about the topic, the more likely that they will know the answer, leading to a higher FOK.
Computationally, an FOK based on cue familiarity must be a function of the cue $Q = \setof{\tuple{p, o}{\in}P{\times}O}$.
We consider two sources that might indicate the ``familiarity'' of the cue: as its activation, or as its connectivity.

\subsubsection{Activation}

One possible metric for the familiarity of the cue is the activation of each individual concept in the cue.
Since activation reflects how recently and frequently a concept has been encountered, concepts with a high activation would be ones that are presented often, which in turn suggests that the agent would be familiar with them.
Formally, this account of FOK could be defined as:
$$\fok = \fok(Q) = f\left(A(o_1), ..., A(o_n)\right) \; \forall {\tuple{p, o}{\in}Q}$$
For the purpose of normalization, this raw value could be placed in the context of the activation of all concepts in LTM.

\subsubsection{Connectivity}

In contrast to activation, connectivity captures how much knowledge the agent has of each component of the cue.
A concept in which an agent is knowledgeable would be connected to many other concepts, while a concept of which the agent is ignorant would only be sparsely connected.
In the extreme, the simple presence or absence of the concept (i.e., whether the agent recognizes the concept) may be a sufficient signal to terminate search, and it has been shown that recognition is can be a useful heuristic for knowledge search \cite{Li2012FunctionalInteractionsBetween}.

The connectivity of a concept is measured by its \textit{fan}, the number of incoming (fan-in) and outgoing (fan-out) edges.
Arguments could be made for only considering fan-in or fan-out.
The fan-in would represent \fixme{FIXME}, while the fan-out would represent \fixme{FIXME}.
Of course, it is also possible to consider the overall fan of a concept, regardless of the direction of the edges.
We leave these implementation details as future work, and leave the precise meaning of the $\text{fan}(s{\in}S)$ function undefined.
Formally, this account of FOK could be defined as:
$$\fok = \fok(Q) = f\left(\text{fan}(o_1), ..., \text{fan}(o_n)\right) \; \forall {\tuple{p, o}{\in}Q}$$
As with an FOK based on activation, this raw value could be normalized within the distribution of fan values of all concepts in LTM.

% FIXME talk about spreading as a combination of activation and connectivity
%Second, a broader sense of connectivity may include not only the immediate neighbors of the concept, but neighbors up to some graph distance away (whose weight may be attenuated based on the inverse of that distance).

\subsection{Accessibility}

Unlike cue familiarity metrics which depend on the cue, the accessibility account of FOK considers information that is only available during and after a retrieval, using the ``byproducts`` of the retrieval process \cite{Koriat1993HowDoWe}.
The intuition behind the accessibility account is that the retrieval process may provide hints as to whether the agent could answer the question; if the first result or the metadata is one with low confidence or high uncertainty, this may result in a low FOK.
For the purpose of this paper, we expand on the idea of accessibility to also include properties of the result of a retrieval as well as metadata from the process.

\subsubsection{Activation}

As with the activation of the cue, the activation of the result of a retrieval may be a source of FOK.
Beyond summarizing the recency and frequency of use and therefore whether the concept is familiar, activation in this context may also represent the speed of the retrieval: the higher the activation, the more quickly the retrieval occurs \cite{Anderson2004AnIntegratedTheory}.
There is a large literature on the correlation between fluency and various memory phenomena \cite{Alter2009UnitingTheTribes}, but here we consider it as equivalent to the activation of the retrieved result under the common model.

Formally, this account of FOK could be defined as:
$$\fok = \fok(s) = f(A(s))$$
where $s$ is the result of a retrieval.
While this formulation may not be meaningful for architectures that always retrieve the most highly activated entity that matches the query, the introduction of noise or other ranking metrics may lead to different FOK behavior.

Here the normalization group \fixme{has more options}.
Beyond using the activation of all concepts in LTM, another possible comparison group is against all candidates for the retrieval --- that is, the activation of other concepts that match the retrieval cue.
The intuition is that the agent is evaluating the relative importance of this result against other potential results.
Such a comparison group would blur the difference between the accessibility account with the competition account, which we discuss in the next section.

\subsubsection{Connectivity}

A different source of information about the result of a retrieval is its connectivity, or the number of graph neighbors it has.
As before, the connectivity of an entity represents the amount of knowledge that the agent has about the result.
Note that the activation of an element and its connectivity are not necessarily correlated.
While spreading activation may cause better-connected entities to have higher activation, it is possible for a concept to be well-understood (i.e., have high connectivity) but irrelevant to the recent/current situation (i.e., have low activation), as is the case when false memories are induced \cite{Li2016TowardsModelingFalse}.

Formally, this account of FOK could be defined as:
$$\fok = \fok(s) = \ln(\frac{\text{fan}(s)}{\overline{\text{fan}}_S})$$
As before, this raw value could be normalized against all concepts in LTM, or only against other candidates for the retrieval.

\subsubsection{Other} % FIXME 

Given the broad definition of the accessibility account, other metadata of the retrieval process and other properties of the results have been proposed, although they do not map as cleaning onto the existing memory mechanisms of the common model.
One such possibility is for FOK to be based on a \textit{partial retrieval} from LTM, where an agent retrieves some but not all knowledge about a concept \cite{Hanczakowski2017MetamemoryInA}.
The intuition is that a partial retrieval suggests to the agent that a complete retrieval is possible, thus leading to an FOK.
While this theory is psychologically plausible, we note that partial retrievals are not currently a feature of ACT-R, and designing such a mechanism for the common model is outside the scope of this paper.

Similarly, incorrect information about the target may contribute to FOK as well \cite{Koriat1993HowDoWe}.
It is unclear, however, whether this is distinct from when the correct answer is retrieved, since the agent has no a priori knowledge that the retrieved result is wrong.
It's possible that the higher FOK from an incorrect result is due to the agent \textit{believing} it has the answer, the same as if the answer was in fact incorrect.
For this reason, we ignore the correctness of a retrieval, and do not consider FOK from an incorrect result any differently from other results.
We do note that, given the dynamic account of FOK, most of the results from intermediate queries will not be an answer to the original question, and thus may be considered ``incorrect'' results. % FIXME

Finally, the semantic content of the result itself may be a source of FOK.
The agent may be able to infer the correct answer from the result of a retrieval, and performing this inference may boost the FOK.
The inference process is highly dependent on the question being asked as well as the existing knowledge and capabilities of the agent, however, and given the large space of possibilities for both aspects, we leave the relationship between semantic content and FOK for future work.

\subsection{Competition}

Less commonly discussed than the cue familiarity and accessibility accounts of FOK is the competition account \cite{Schreiber1998TheRelationBetween}.
The main distinction of this account is that it is not the cue or the result of a retrieval that determines FOK, but rather the \textit{candidates} for a particular query.
The intuition for the competition account is that, the more potential results to a query, the lower the FOK should be, with the intuition being that the agent is unsure which of the possible results is correct.
A direct translation of the competition account is to use the inverse of the number of candidates, which could be defined as:
$$\fok = \fok(Q) = \frac{1}{|S_Q|}$$

We could, however, broaden the idea of using the candidates to determine FOK, by considering the distribution of the activation or connectivity of the candidates.
Much like how a large number of retrieval candidates could indicate uncertainty, so could a uniform distribution of the activations of the candidates.
Such a distribution would indicate that no candidate is more likely than the other, hence suggesting uncertainty; in contrast, a peaked distribution would mean that the candidate with more probability mass is likely to be the correct answer.
An activation-based competition account of FOK could be defined as:
$$\fok = \fok(Q) = \frac{1}{\text{Var}(\{A(s) \forall s \in S_Q\})}$$ % FIXME
As with other accounts, the variance of the activation values could be replaced with other summary statistics such as the standard deviation, as long as it correlated variance and inversely correlated with the uniformity of the activation values.

We note that while the competition account of FOK may technically be a subcategory of the accessibility account, in that the number of candidates is a piece of metadata from the retrieval process, we consider the competition account sufficiently different to address it separately.
In particular, unlike the accessibility account where the result of a retrieval plays a main role, the competition account (in the extreme) does not consider the result at all.
Rather, it is the distribution of the activation or connectivity of the candidates that influence FOK; which specific concept is retrieved does not play a role.

\subsection{Other Accounts of FOK}

Cue familiarity, accessibility, and competition represent the main accounts of FOK, but other theoretical mechanisms are possible \cite{Nelson1984AccuracyOfFeeling}.
Of particular note is how agent's expectations about the question may influence FOK, as demonstrated by \textcite{Widner1996TheEffectsOf}, where subjects were told that questions were either ``easy'' or ``difficult''.
Another possible source of information for FOK is from episodic memory, as demonstrated by \textcite{Schwartz2014ContextualInformationInfluences}, in contrast to the assumption that LTM refers to semantic memory.
These two sources may be combined in cases where the question asks for knowledge from a national curriculum (as in the US TV show \textit{Are You Smarter than a 5th Grader?}), which might lead to an expectation that the answer is easy, as well as an episodic memory that the answer was known at some point in the past \cite{Singer2008FeelingOfKnowing}.
Such complex metamemory manipulations are outside the scope of this paper, although their connections to FOK may be a fruitful future direction of research.

\subsection{Hybrid Accounts}

The accounts of FOK listed in this section thus far are individual sources, meaning that FOK considered exclusively a function of one type of memory metadata.
In practice, FOK may be the result of more complex combinations of some or all of these sources, that together lead to the FOK that people report.
This idea is not new, as it has been noted that cue familiarity is available after the question is asked but before a retrieval, while accessibility is only available during or after a retrieval.
It has therefore been suggested that these could be used sequentially: that FOKs solicited earlier are a result of cue familiarity, and FOKs solicited later are a result of accessibility \cite{Florer2000FeelingsOfKnowing,Koriat2001TheCombinedContributions}.

Mathematically, other \fixme{functions} of FOK are possible as well.
Although we have considered activation and connectivity as separate dimensions of measurement, an FOK could use both.
For example, instead of using the fan of a retrieval result directly, FOK could be calculated by weighting each edge by the activation of the neighboring entity, resulting in an FOK that combines more information about the agent's knowledge.
Such a calculation is reminiscent of spreading activation, suggesting that it is psychologically plausible.
A systematic exploration of ways to combine sources of FOK, and what they would mean psychologically, is beyond the scope of this paper.

\section{Accounts for a Dynamic FOK}

As mentioned earlier in this paper, the psychology literature has focused on FOK as a single measurement during the process of question answering.
The sources of FOK discussed thus far have come from this tradition, although the hybrid accounts hint at how different sources might be combined. 
We now return to the idea that FOK may instead by dynamic, changing over time as different strategies are used to answer a single question.
This section considers the ramifications of this hypothesis, and proposes additional possibilities for how FOK may be determined.

% what does this mean for existing psychology experiments?
We first note that the FOK solicited in past experiments are likely not invalid, but they are likely to only provide a narrow view of FOK.
These measurements may only be accurate to the state of knowledge search of the subject at the time of solicitation, and without a detailed understanding of the search state, it is difficult to isolate how the FOK was generated.
Even assuming the cue familiarity or accessibility accounts, it raises questions as to what cues were used for familiarity judgments, or what retrieval metadata were used for when accessibility is measured.
The possibility of multiple retrievals that occur in sequence also muddle the distinction between retrieval cues and retrieval results, since the result of one retrieval may become the cue for the next retrieval.
This in fact makes the hybrid account of FOK more parsimonious, as it is not that different sources are used in at different points of problem solving, but that the targets \textit{becomes} cues as multiple retrievals occur.

% what does this mean for existing sources?
A dynamic FOK has implications not just for which sources dominate (if they are indeed different sources at all), but what information each source provides.
During the course of problem solving, the activation of entities will change based on the results of previous retrievals.
A cue that initially had low activation may be boosted if multiple retrieval results are connected to it; conversely, previously highly activated entities may become less so over time.
While the connectivity of LTM is less affected by retrievals, it is also not impossible that new connections could be made during problem solving, for example if an agent realizes that blue whales are not fish in answering \question{What is the largest fish on earth?}{Whale sharks}.
In sum, the sources do not only provide a single value, but a \textit{history} of values which could be combined into an FOK judgment.

% what other sources might this allow for?
Intriguingly, access to the history of sources expands \fixme{range} of inputs for determining FOK, notably that of \textit{previous} FOK values.
Consider again the question of what is the capital of Australia, and where a subject might guess with several large cities before giving up.
Although this could be explained by the accessibility account of FOK --- where more prominent cities such as Sydney are given first, before less-well-known cities like Perth --- another plausible model is that the search termination is not due to just the activation of the last retrieved concept, but due to the overall downward trend of activation.
In this case, the FOK judgments are not based purely on activation, but is additionally modulated by how the FOK it self has changed over time.
Mathematically, we might define FOK to be a function of time, $\fok_t$, with $t$ being the number of steps in the past.
In this example, the fact that $\fok_{{\negsub}3} > \fok_{{\negsub}2} > \fok_{{\negsub}1}$ would further decreases the FOK judgment.
Beyond the trend of previous FOKs, two other plausible functions may be a weighted average of previous FOKs, or simply the number of retrievals that have occurred (as an indication of whether the agent has exhausted possible answers).
More generally, FOK could be defined as
$$\fok = f(\fok_-1, ..., \fok_T) = $$
up to some time $T$ in the past, plus additional \fixme{parameters} corresponding to the cue familiarity and accessibility accounts.

% what difficulties does this present for cognitive modeling?
\comment{
    mathematical pitfalls
        repeated retrievals will boost activation; 
    the impact of "unrelated" retrievals/inferences
    any modeling issues?
}

% conclusion?

\section{Evaluating Models of Feelings of Knowing}

% FIXME what does a "high" FOK mean? have been silent thus far on what values FOKs could have
% Without loss of generality, we further define the FOK to be a real number between -1 and 1; this allows the agent to understand whether an FOK is high or low in an absolute sense, without \fixme[I'm not sure how that would work]{the need for comparison against other FOKs}.

Despite the plethora of plausible sources of FOK, it is difficult to empirically evaluate whether these models reflect how FOK is computed.
Especially with the possibility of a dynamic FOK that changes with each memory retrieval, there is currently insufficient human data to determine how FOK changes over time.
It is unclear that FOK should always increase until an answer is found, nor that it should always decrease if the agent eventually gives up.
For example, one could try a strategy that fails to produce an answer, before switching strategies and succeeding; alternately, a strategy may be initially promising, before running into a dead end.
% FIXME need to say more here - to what extent might FOK be co-learned with memory strategies?
% that is, any particular memory strategy must be matched with an appropriate FOK
More generally, the interaction between FOK and the other strategic aspects of memory retrieval remains unexplored, leaving a lot of uncertainty as to how FOK changes over time.
Nonetheless, we propose that there are two main approaches to evaluation: to match against human data, or to show that the computed FOK is empirically useful in a broader cognitive model.
We discuss each type of evaluation below.

\subsection{Evaluation via Matching Human Data}

The majority of experimental results on FOK fall into one of two categories.
The first type of experiments measure how different factors affect FOK, and whether it increases or decreases the judgment. % FIXME citation examples?
The second type of experiments measures the accuracy of FOK and the degree to which it correlates with whether the subject retrieves the correct answer, where the accuracy may also depend on the same factors. % FIXME citation examples?
One challenge in matching human data is that, with the dynamic view of FOK, it is unclear at which point FOK is measure.
That is, if four retrievals are needed to answer a question, the FOK will be different after every retrieval regardless of which source is used.
% FIXME conclusion

% FIXME perhaps what needs to happen here is a table:
% {cue, target, competition} X {activation, connectivity, other}

A good summary of factors that influence FOK is given in \textcite{Schwartz1994SourcesOfInformation}.
The factors themselves map well onto either changing the activation of some elements, changing the connectivity of LTM, or some other manipulation.
For example, the experiments in \textcite{Metcalfe1993TheCueFamiliarity} and \textcite{Schwartz2014ContextualInformationInfluences} both manipulate FOK by varying the amount of information presented.
Since the information was artificially created by the experimenters, they do not correspond to any existing concepts in LTM; these manipulations therefore change the connectivity of LTM.
We might therefore expect connectivity-based FOK models of FOK to be affected, and thus can evaluating whether the direction of change matches that in people.
In contrast, \textcite{Florer2000FeelingsOfKnowing} examine the effects of repeated elements (within an artificial sequence).
Once the sequence has been encoded as structure in LTM, the repetition would increase the activation of those elements; these manip[ulations therefore change the activation of LTM.

% experiments/factors that affect both activation and connectivity
% domain familiarity (Schwartz1994SourcesOfInformation)

\subsection{Evaluation via Functional Demonstration}

% functional evaluation
The other main approach to evaluating FOK is examine whether it is a valid signal for an agent knowing the answer to a question.
Unlike matching human data, the focus here is simply on the ``accuracy'' of FOK, as suggested by the cognitive-heuristic account of metamemory.
% FIXME more AI-like

% looking at how FOKs differ in "performance" under different conditions
% unclear what "performance" means, if FOK does not monotonically increase or decrease

\comment{
    general issues: highly dependent on existing knowledge of agent and questions asked
        due to activation levels, inferential ability, etc.
        could focus on fact-based questions, eg. capital of Australia
            however, could still have facts that require inference; see Jeopardy questions
    how to evaluate computational FOK?
        evidence that FOK is correlated with the answer being correct FIXME citation needed
        what about matching human data?
            current memory systems unable to do so \cite{Li2016TowardsModelingFalse}
            issues with sufficient knowledge and inference mechanisms
    dynamicism adds additional challenges
        unclear what FOK should look like over time
        not clear that it should always increase, for example
            why not?
            questions about whether FOK offers directionality, or can be used to calculate directionality
                if no, suggests neighborhood metrics may not be correct
                however, may not be psychologically valid
        but without
    dynamic FOK requires a Q&A setting
        need external source to confirm incorrect answer
    could flip this around - use existing agents and measure what different FOKs look like
        with the goal of evaluating which ones lead to successful retrievals
        there's something weird here in that we want to look at cases where the agent fails as well
    modeling/data match vs. functional evaluation
}

% Koriat1993HowDoWe has a section on explaining the accuracy of FOK

%For relative activation, outgoing edges, activation over edges 2, competition 2, and absolute activation FOKs, calculations were made based on the cue for queries and the target for retrievals.
%For the historic FOK calculation, we first calculated the weighted average of past FOKs (gamma of 0.48), found the new differences between the most recent FOK and the previous one, then took another weighted averages of these new differences. (The comments in the code are unclear and some of it looks wrong? So it might be worth looking at again. I also have no idea why we chose a gamma of 0.48)
%we used different FOK calculations based on where in the retrieval process the FOK occurred (using one method for a query and another method for subsequent retrievals)

\section{General Discussion}


% PLACEHOLDER TO TRIGGER CITATIONS
%
% \cite{Burgess1996ConfabulationAndThe}
% \cite{DeSoto2014PositiveAndNegative}
% \cite{Florer2000FeelingsOfKnowing}
% \cite{Hanczakowski2014FeelingOfKnowing}
% \cite{Hanczakowski2017MetamemoryInA}
% \cite{Koriat1993HowDoWe}
% \cite{Koriat2001TheCombinedContributions}
% \cite{Li2012FunctionalInteractionsBetween}
% \cite{Li2016ArchitecturalMechanismsFor}
% \cite{Mangan2000WhatFeelingIs}
% \cite{Nelson1984AccuracyOfFeeling}
% \cite{Nelson1990MetamemoryATheoretical}
% \cite{Nelson2008TowardsARational}
% \cite{Norman2016TheRelationshipBetween}
% \cite{Salvucci2015EndowingACognitive}
% \cite{Schreiber1998TheRelationBetween}
% \cite{Schwartz1994SourcesOfInformation}
% \cite{Schwartz2011TipOfThe}
% \cite{Schwartz2014ContextualInformationInfluences}
% \cite{Sharma2016ControllingSearchIn}
% \cite{Widner1996TheEffectsOf}

\bibliographystyle{apacite}

\setlength{\bibleftmargin}{.125in}
\setlength{\bibindent}{-\bibleftmargin}

\bibliography{iccm}

\end{document}
